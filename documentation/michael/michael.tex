\documentclass[paper=a4,fontsize=11pt,twocolumn,pagesize,bibtotoc]{scrartcl}

\usepackage[utf8]{inputenc}
\usepackage[T1]{fontenc}
\usepackage[english]{babel}

\usepackage{tikz}
\usepackage{amssymb,amsmath}
\usepackage{graphicx}
\usepackage{color}

\usepackage{listings}

% Customize the appearance of code listings:
\lstset{
  backgroundcolor=\color{white},
  basicstyle=\small\ttfamily,
  breakatwhitespace=false,
  breaklines=true,
  captionpos=b,
  commentstyle=\color{green},
  frame=single,
  keepspaces=true,
  keywordstyle=\color{blue},
  language=C,
  numbers=left,
  numbersep=5pt,
  numberstyle=\tiny\color{gray},
  rulecolor=\color{black},
  showspaces=false,
  showstringspaces=false,
  showtabs=false,
  stepnumber=1,
  stringstyle=\color{mauve},
  tabsize=2,
  title=\lstname
}

\usepackage{hyperref}

% Customize the appearance of hyperlinks and of the pdf reader:
\hypersetup{
  colorlinks=true, % removes border, allows text coloring
  linkcolor=blue, % color of in-document links.
  urlcolor=blue, % color for weblinks/email
  pdfborderstyle={/S/U/W 1} % 1pt underline (instead of a box)
}

% Remove the annoying ACM copyright notice:
\makeatletter
\def\@copyrightspace{\relax}
\makeatother

\title{CMIDID Kernel Driver\\Module Setup and GPIO Input}
\subtitle{Praktikum Linux \& C - Technische Universität München}

\author{Michael Opitz}

\begin{document}
\maketitle

\section{Introduction}
\label{michael:introduction}

In this part of the documentation, the setup of the CMIDID driver and the 
distribution into several components will be explained, as well as the 
implementation of the GPIO component, which is the one that handles the input 
on GPIO ports and translates that input into MIDI events.
I spent most of my time during the 
development phase working on the latter component, but before we actually 
started to implement the driver, I looked into the high-resolution timer API 
of the kernel. That was because we knew from the beginning that measuring the 
time difference between consecutive key hits (of our breadboard MIDI keyboard) 
would be required. Fortunately, the hrtimer API turned out to be very 
userfriendly and in the end we only required some simple functions like 
\begin{lstlisting}
ktime_get();
hrtimer_start(struct hrtimer *timer, ktime_t tim, const enum hrtimer_mode mode);
hrtimer_init(struct hrtimer *timer, clockid_t which_clock, enum hrtimer_mode mode);
\end{lstlisting}
and a couple more (see: \url{https://github.com/torvalds/linux/blob/master/include/linux/hrtimer.h}).
So, we decided that we don't need a separate module 
component to handle the timer related functionality and included this in the 
GPIO component.

So, besides the timer component, which was not required, and the GPIO 
component I already mentioned, we did add a main component, which handles 
initialization of the other components, as well as a MIDI component, which is 
responsible for creating a virtual MIDI device, and for outputting MIDI 
events on that device.

In the section \textbf{\nameref{michael:setup}}, I'm going to show how the compilation of 
the several module components works and how the component initialization and 
exit routines have to be handled. The other part of his documentation is 
a detailed explanation of the GPIO component: \textbf{\nameref{michael:gpios}}. This 
includes the steps necessary to read input on the GPIO ports from the kernel 
module, a description of a clean exit routine, as well as a couple of special 
cases and workaround for bugs, that are worth mentioning. The GPIO component 
has a significant amount of code for handling module customization via 
\texttt{ioctl} and module parameters; this won't be explained in this part of 
the documentation, but rather in (TODO: CITE JANNIK). And for an indepth guide 
on the MIDI component, take a look at (TODO: CITE ANDI).

\section{Project Setup}
\label{michael:setup}

The CMIDID kernel driver found in the \texttt{module} subdirectory, is split
up into the following components. The following enumeration and the following
subsections contain a short overview for each of those components as well as 
short explanation of our decision as to why we needed/created those components.
\begin{itemize}
  \item The main component: \texttt{cmidid\_main.c}. That's the one that 
    handles the actual module initialization and the module cleanup, aw well 
    as the creation of a character device for \texttt{ioctl}, and the 
    \texttt{ioctl} callback function. See section \textbf{\nameref{component:main}}.
  \item The MIDI component: \texttt{cmidid\_midi.c}. This component handles 
    the creation of the virtual MIDI soundcard via the ALSA sequencer 
    interface, and the sending of MIDI events (in our case only the events 
    \texttt{noteon} and \texttt{noteoff}). See section \textbf{\nameref{component:midi}}.
  \item The GPIO component: \texttt{cmidid\_gpio.c}. This component is 
    responsible for reading input on the GPIO ports, for translating the 
    input into MIDI keyboard key hit events and for informing the MIDI 
    component to actually send a MIDI event. See section \textbf{\nameref{component:gpio}}.
\end{itemize}
In-tree modules are usually contained in a signle \texttt{.c} file, but we 
used this setup to make a clean seperation of the individual logical parts of 
the module and to increase readability of each single source file. We added 
a header for every single component, where the \texttt{cmidid\_ioctl.h} header 
is special, in the way that it needs to be included by userspace programs that
want to communicate with the module via \texttt{ioctl}. And the \texttt{cmidid\_util.h}
header contains only a couple of \texttt{printk} macros to ease the output of 
debug information into the kernel log. The final structure of our project 
setup looks like this:
\begin{lstlisting}
module/
  Makefile
  cmidid_main.h
  cmiddi_main.c
  cmidid_gpio.h
  cmidid_gpio.c
  cmidid_midi.h
  cmidid_midi.c
  cmidid_ioctl.h
  cmidid_util.h
\end{lstlisting}
Additional to following overviews of each component, the next section provides
a brieft introduciton of the KBuild system and how we used this to compile 
the components into a single module.

\subsection{Compilation with Kbuild}
\label{component:compilation}

The Linux kernel uses a system named Kbuild which is thoroughly described in 
\url{https://github.com/torvalds/linux/blob/master/Documentation/kbuild/}.
For building external (i.e. out-of-tree) modules, the documentation file
\url{https://github.com/torvalds/linux/blob/master/Documentation/kbuild/modules.txt}
was especially relevant and explained the creation of makefiles for this type
of module.

For compiling a single out-of-tree module that consists of several components,
one can provide the name of the module and the names of the components with
the following syntax to Kbuild:
\begin{lstlisting}[language=make]
obj-m := cmidid.o
cmidid-objs := cmidid_midi.o cmidid_main.o cmidid_gpio.o
\end{lstlisting}
Kbuild will derive the required source files from the given information.
The only thing left to do is to build the actual module with
\begin{lstlisting}[language=make]
all:
	$(MAKE) -C $(KSRC) M=$$PWD modules
\end{lstlisting}
The \texttt{-C} flag tells \texttt{make} to change in the directory containing the 
sources for the running kernel, and the \texttt{M} option is required for the 
location of the external module that we are compiling.

\subsection{The Main Module Component}
\label{component:main}

The component \texttt{cmidid\_main.c} of the CMIDID module does the following:
\begin{itemize}
  \item Setting up the required module information, like the init and exit 
    function with the \texttt{module\_init} and \texttt{module\_exit} macros.
  \item Creation of a character device in the initialization routine. This
    device is required for \texttt{ioctl} and it's created with the \texttt{cdev}
    interface. First the character device needs to be created with the 
    subsequent calls \texttt{alloc\_chrdev\_region} (gets the device number), 
    \texttt{cdev\_alloc} (allocates space for the actual device) and
    \texttt{cdev\_add} (adds the device to the system). Finally, for the character
    device to appear in the sysfs filesystem, we need can create the device 
    node from our module with \texttt{device\_create}.
  \item Handling of the initialization routine for every other module 
    component. There's one problem that needs to be taken care of: If the 
    GPIO component is initialized after the MIDI component, and if the 
    initialization of the MIDI component fails, we can't just call \texttt{exit(-1)}
    to terminate the module. Every memory allocate in the MIDI init routine
    needs to be freed manually, so that we can exit without memory leaks.
    The most readable way to do this, is to call the cleanup functions in 
    inverse order via a list of \texttt{goto} labels like this:
\begin{lstlisting}[language=C]
if ((err = cmidid_midi_init() < 0)
  goto err_midi_init;

if ((err = cmidid_gpio_init( ) < 0)
  gotot err_gpio_init;

err_gpio_init:
  cmidid_midi_exit();

err_midi_init:
  free character device
  ...
\end{lstlisting}
  \item The \texttt{exit} routine of the module needs to call the subcomponent
    cleanup routines in a similar fashion to guarantee a clean exit.
  \item Additional to the above, the main component also handles the \texttt{ioctl}
    user input. This process is explained more detailed in Jannik's part of the 
    documentation.
\end{itemize}

\subsection{The MIDI Module Component}
\label{component:midi}

\subsection{The GPIO Component}
\label{component:gpio}

\section{Working with GPIOs}
\label{michael:gpios}
In this section, the GPIO component of the CMIDID driver is described in more
detail. The component consists for the first part of the header \texttt{cmidid\_gpio.h} which
provides several function definitions for the main module, including the 
init and exit routines of the GPIO component, and functions for \texttt{ioctl}
configuration. The corresponding source file is \texttt{cmidid\_gpio.c}.
The following subsections describe some of the more interesing parts of the 
component. A definition of the GPIO interface of the kernel can be found in 
the header file \url{https://github.com/torvalds/linux/blob/master/include/linux/gpio.h}.

\subsection{The \texttt{key} Struct}
\label{gpios:key}

The main abstraction that is used in the GPIO component, is the \texttt{struct key}
which describes an abstract key of a MIDI keyboard. Here's the definition:
\begin{lstlisting}
struct key {
  KEY_STATE state;
  struct gpio gpios[2];
  unsigned int irqs[2];
  ktime_t hit_time;
  bool timer_started;
  struct hrtimer hrtimers[2];
  unsigned char note;
  int last_velocity;
};
\end{lstlisting}
Each key of our keyboard has two buttons which are required to determine the
velocity/force of the key hit, so we need a pair of GPIO ports for every button.
Most of the other struct members are necessary to actually calculate the 
hit velocity for the specific key and the \texttt{note} member associates a
MIDI note with the button.

\subsection{GPIO Interrupts}
\label{gpios:interrupts}
Fortunately, the Raspberry Pi supported interrupts for every GPIO pin, so it
was not necessary to write extra code to listen on each port for rising and 
falling edge triggers. We could rather specify a callback function for every
IRQ: \texttt{irq\_handler} in \texttt{cmidid\_gpio.c}. That's the function we
used to calculate the actual time difference between the button hits for each
keyboard key.

\subsection{Initialization}
\label{gpios:init}

The GPIO initialization function \texttt{cmidid\_gpio\_init} takes care of of 
several required steps to setup working GPIO ports.

\begin{itemize}
  \item The number of required GPIOs is determined via the \texttt{gpio\_mapping}
    module parameter.
  \item It's necessary to request the GPIO port numbers which need to used.
    The most readable way to do this, is to define an array of \texttt{gpio}
    structs and call the function \texttt{gpio\_request\_array}. (In the case
    of an error those need to be freed with \texttt{gpio\_free\_array}.)
  \item The high resolution timers, which are required for the time 
    measurement, need to be initialized and the callback functions for each
    timer is set to \texttt{timer\_irq}.
  \item The IRQs must be requested, so that the \texttt{irq\_handler} can be 
    called on an interrupt. This is achieved by getting an IRQ number for 
    every GPIO with the \texttt{gpio\_to\_irq} routine. (That part is 
    architecture specific and not necessarly available on every ARM system.
    Fortunatley, it's available on the Raspberry Pi.) After getting the IRQ
    number, \texttt{request\_irq} needs to be called, finally.
\end{itemize}

\subsection{The IRQ Handler}
\label{gpios:irqhandler}

After successfully initializing the GPIOs and IRQs, the only thing necessary 
to define is the interrupt callback function \texttt{irq\_handler}. The function
signature is:
\begin{lstlisting}
  irqreturn_t irq_handler(int irq, void *dev_id);
\end{lstlisting}
So, with our setup, the each key struct is associated with two irq numbers, 
one for the first and one for the second button. We can set the touch and the 
hit time time for each key when the handler is called with one of the associated IRQ
numbers. 

One important consideration here is, that we need to take care of possible 
button jittering manually. We solved this problem by calling another callback
function which is delay with a high resolution timer. Until this second 
callback function is executed, we ignore every subsequent input on the 
current GPIO port, which seems to be a simple, yet reliable jitter resolution.

\end{document}

