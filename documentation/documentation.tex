% documentation.tex
%
% Using the Alternate ACM Sig Proceedings style.
% Get it at:
%   http://www.acm.org/sigs/publications/proceedings-templates
%

\documentclass{sig-alternate}

\usepackage[utf8]{inputenc}
\usepackage[english]{babel}

\usepackage{tikz}
\usepackage{amssymb,amsmath}
\usepackage{graphicx}
\usepackage{color}

\usepackage{listings}

% Customize the appearance of code listings:
\lstset{
  basicstyle=\ttfamily,
  numbers=none,
  stepnumber=5,
  tabsize=2,
  firstnumber=1,
  numberfirstline=false,
  numberstyle=\ttfamily,
  frame=single,
}

\usepackage{hyperref}

% Customize the appearance of hyperlinks and of the pdf reader:
\hypersetup{
  colorlinks=true, % removes border, allows text coloring
  urlcolor=blue, % color for weblinks/email
  pdfborderstyle={/S/U/W 1} % 1pt underline (instead of a box)
}

% Remove the annoying ACM copyright notice:
\makeatletter
\def\@copyrightspace{\relax}
\makeatother

\begin{document}

\title{CMIDID Kernel Driver \& Linux AppleMIDI\\Documentation}
\subtitle{Praktikum Linux \& C - Technische Universität München}

\numberofauthors{4}

\author{
\alignauthor
Felix Engelmann\email{\texttt{felix.engelmann@tum.de}}
\alignauthor
Michael Opitz\email{\texttt{opitz@in.tum.de}}
\alignauthor
Andreas Ruhland\email{\texttt{ruhland@in.tum.de}}
\and
\alignauthor
Jannik Theiß\email{\texttt{jannik.theiß@tum.de}}
}

\maketitle

\section{Overview}
\label{documentation:overview}

\subsection{Introduction}
\label{overview:intro}

This document is split up into two main parts: The documentation for the 
\emph{CMIDID} Linux kernel driver and the documentation for the 
\emph{AppleMIDI} kernel driver. Development on both started in the summer 
2014 during the \emph{Linux- und C} internship at \emph{TUM}. The original 
goal of this project was to build a Linux kernel driver which implements 
the \emph{RPT MIDI} standard (as described in 
\href{http://tools.ietf.org/html/rfc6295}{RFC 6295}), yet after some initial 
discussion, the decision was made to focus developement on the CMIDID driver, 
which enabled us to build custom MIDI hardware with low-cost devices like, in 
our case, the \href{www.raspberryp.org}{Raspberry Pi}. Later during the 
development phase, after some promising progress on the CMIDID project, we 
started to implement a altered version of the RTP MIDI driver in the 
form of the \emph{AppleMIDI} driver, which supports only a subset of the 
complete RTP MIDI standard, yet understands the Apple RTP MIDI session 
management - an extension of standard.

The remainder of this document describes the main components and challenges 
of this project in detail. For a short overview of the implemented 
functionality and the possiblities available with CMIDID and Linux AppleMIDI, 
take a look at the project presentation in \texttt{/presentation/} and if 
you are looking for an introduction to build and test the drivers, read the 
\texttt{README.md}.

\subsection{CMIDID Kernel Driver}
\label{overview:cmidid}

The CMIDID Linux kernel driver can be used to build a fake MIDI keyboard 
by translating input on GPIO pins into arbitrary MIDI key events. For our 
demo setup, we used a Raspberry Pi with an attached breadboard to build
several MIDI keys via buttons, resistors and jumper cables. The CMIDID driver 
is highly customizable, and can, for example, use the input on two separate 
GPIOs to simulate a single MIDI key, with the added benefit, that the time 
difference between the GPIO events allows the calculation of the corresponding 
key hit velocity. With that in mind, it's possible to construct more advanced 
MIDI hardware compared to other projects with a similar scope (like, e.g. the 
\emph{Makey Makey} controller @ \url{http://makeymakey.com}).

\subsubsection{Authors}
\label{cmidid:authors}

CMIDID was written by Michael Opitz, Andreas Ruhland and Jannik Theiß in 
collaboration. TODO: ADD SHORT OVERVIEW HOW WE DISTRIBUTED THE WORKLOAD.
I CAN'T REMEMBER EXACTLY...

\subsubsection{Module Structure}
\label{cmidid:structure}

The CMIDID module is split up into several components which was done in 
favor of good separation of concerns. The module is composed of three 
components: The main component, the GPIO component and the MIDI component.
The main component handles mostly the setup and cleanup of the other 
components as well IOCTL. The MIDI component handles the creation and control 
of the corresponding virtual MIDI device and the GPIO component translates 
GPIO input into MIDI events.

LINK TO INDEPTH DISCUSSION OF MODULE STRUCTURE: MICHAEL

\subsubsection{Hardware}
\label{cmidid:hardware}

The CMIDID driver is intended to be run on a Raspberry Pi, which comes with 
several GPIO ports for simple hardware prototyping and which is able to run 
a Linux kernel as well as general purpose applications. In theory, every 
devices which supports Linux and has GPIO pins can be used to replicate our 
setup. Considering that we need to have software for audio processing, e.g.
an Arduino board is not a reasonable choice to build CMDIDID keyboards.

An introduction on the challenges of compling and running CMIDID on the 
Raspberry Pi can be found here:

LINK TO RASPBERRY PI SETUP: ANDI

\subsubsection{Custom Keyboards}
\label{cmidid:keyboards}

Custom MIDI keyboard key events are simulated by interpreting input on the 
GPIO ports as key hits and key releases. This allows flexibility for the 
construction of custom hardware, but has the downside of increased hardware 
prototyping time and more involved driver configuration.

An overview of our sample keyboard setup on a breadboard can be found here:

LINK TO KEY CONSTRUCTION: JANNIK

\subsubsection{Module Configuration}
\label{cmidid:configuration}

The flexiblity of hardware development is probably the greatest advantage of 
the CMIDID module. To simpliy the process of MIDI keyboard construction, we 
added several ways to configure the module and the driver. It's possible to 
pass module parameters to define options which won't change during module 
runtime, like the number of available GPIO pins and the CMIDID driver can 
be configured during runtime via IOCTL from the userspace, which can be used 
for example to transpose the MIDI notes, i.e. add a constant to every note 
event.

The available configuration options are listed in:

LINK TO CMIDID CONFIGURATION: JANNIK

\subsubsection{Virtual MIDI device}
\label{cmidid:midi}

Besides the construction of the keyboard hardware, a virtual software based 
MIDI device is provided to communicate with other sound processing hardware 
like for example a software synthesize like FluidSynth 
(\url{http://www.fluidsynth.org}) or TiMidity++ 
(\url{http://timidity.sourceforge.net}). Fortunatley, the Linux kernel 
offers an interface to easily add MIDI devices to the existing sound 
infrastructure: the ALSA sequencer API. This API offers methods for MIDI 
device creation and manipulation and can be used from the userspace to 
route MIDI events between different devices.

Check out the this section to see how we used the ALSA sequencer API for
our this project:

LINK TO MIDI DEVICE CREATION/ALSA SECTION: ANDI

\subsubsection{GPIO Handling}
\label{cmidid:gpio}

The GPIO component of CMIDID module handles incoming triggers on the GPIO 
pins. Depending on the configuration, rising edge and falling edge events 
are interpreted as MIDI key hits and subsequent releases. Those signals are 
translated into a specific MIDI event: Either noteon or noteoff. Additionally, 
the GPIO component is responsible for the calculation of the velocity for 
the noteon events. This is done by measuring the time difference between 
the virtual key hits and interpolating the velocity according to a given 
interpolation function.

For more details on the handling of GPIO input, see:

LINK TO GPIO SECTION: MICHAEL

\subsection{AppleMIDI}
\label{overview:applemidi}



\subsubsection{Authors}
\label{applemidi:authors}

\end{document}

