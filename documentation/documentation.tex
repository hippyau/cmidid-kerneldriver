\documentclass[paper=a4,fontsize=11pt,twocolumn,pagesize,bibtotoc]{scrartcl}

\usepackage[utf8]{inputenc}
\usepackage[T1]{fontenc}
\usepackage[english]{babel}

\usepackage[osf]{mathpazo}
\usepackage{microtype}
\usepackage{tikz}
\usetikzlibrary{automata}
\usepackage{amsthm, amssymb, amsmath}
\usepackage{graphicx}

\usepackage{color}
\usepackage{xr-hyper}
\usepackage{hyperref}

\parindent0pt

\usepackage{listings}

% Customize the appearance of code listings:
\lstset{
  basicstyle=\ttfamily,
  numbers=none,
  stepnumber=5,
  tabsize=2,
  firstnumber=1,
  numberfirstline=false,
  numberstyle=\ttfamily,
  frame=single,
}

\usepackage{hyperref}

% Customize the appearance of hyperlinks and of the pdf reader:
\hypersetup{
  colorlinks=true, % removes border, allows text coloring
  urlcolor=blue, % color for weblinks/email
  pdfborderstyle={/S/U/W 1} % 1pt underline (instead of a box)
}

% Remove the annoying ACM copyright notice:
\makeatletter
\def\@copyrightspace{\relax}
\makeatother


\title{CMIDID Kernel Driver \& Linux AppleMIDI\\Documentation}
\subtitle{Praktikum Linux \& C - Technische Universität München}

%\numberofauthors{4}

\author{
Felix Engelmann\\ \texttt{felix.engelmann@tum.de}
\and
Michael Opitz\\ \texttt{opitz@in.tum.de}
\and
Andreas Ruhland\\ \texttt{ruhland@in.tum.de}
\and
Jannik Theiß\\ \texttt{jannik.theiß@tum.de}
}

\begin{document}
\maketitle

\section{Overview}
\label{documentation:overview}

\subsection{Introduction}
\label{overview:intro}

This document is split up into two main parts: The documentation for the 
\emph{CMIDID} Linux kernel driver and the documentation for the 
\emph{AppleMIDI} kernel driver. Development on both started in the summer 
2014 during the \emph{Linux- und C} lab at \emph{TUM}. The original 
goal of this project was to build a Linux kernel driver which implements 
the \emph{RPT MIDI} standard (as described in 
\href{http://tools.ietf.org/html/rfc6295}{RFC 6295}), yet after some initial 
discussion, the decision was made to focus development on the CMIDID driver, 
which enabled us to build custom MIDI hardware with low-cost devices like, in 
our case, the \href{www.raspberrypi.org}{Raspberry Pi}. Later during the 
development phase, after some promising progress on the CMIDID project, we 
started to implement a altered version of the RTP MIDI driver in the 
form of the \emph{AppleMIDI} driver, which supports only a subset of the 
complete RTP MIDI standard, yet understands the Apple RTP MIDI session 
management - an extension of the standard.

The remainder of this document describes the main components and challenges 
of this project in detail. For a short overview of the implemented 
functionality and the possibilities available with CMIDID and Linux AppleMIDI, 
take a look at the project presentation in \texttt{/presentation/} and if 
you are looking for an introduction to build and test the drivers, read the 
\texttt{README.md}.

\subsection{CMIDID Kernel Driver}
\label{overview:cmidid}

The CMIDID Linux kernel driver can be used to build a fake MIDI keyboard 
by translating input on GPIO pins into arbitrary MIDI key events. For our 
demo setup, we used a Raspberry Pi with an attached breadboard to build
several MIDI keys via buttons, resistors and jumper cables. The CMIDID driver 
is highly customizable, and can, for example, use the input on two separate 
GPIOs to simulate a single MIDI key, with the added benefit, that the time 
difference between the GPIO events allows the calculation of the corresponding 
key hit velocity. With that in mind, it's possible to construct more advanced 
MIDI hardware compared to other projects with a similar scope (like, e.g. the 
\emph{Makey Makey} controller @ \url{http://makeymakey.com}).

\subsubsection{Authors}
\label{cmidid:authors}

CMIDID was written by Michael Opitz, Andreas Ruhland and Jannik Theiß in 
collaboration.

\subsubsection{Module Structure}
\label{cmidid:structure}

The CMIDID module is split up into several components which was done in 
favor of good separation of concerns. The module is composed of three 
components: The main component, the GPIO component and the MIDI component.
The main component handles mostly the setup and cleanup of the other 
components as well IOCTL. The MIDI component handles the creation and control 
of the corresponding virtual MIDI device and the GPIO component translates 
GPIO input into MIDI events. For a indepth documentation on this topic, take 
a look at \texttt{michael.pdf}.

\subsubsection{Hardware}
\label{cmidid:hardware}

The CMIDID driver is intended to be run on a Raspberry Pi, which comes with 
several GPIO ports for simple hardware prototyping and which is able to run 
a Linux kernel as well as general purpose applications. In theory, every 
devices which supports Linux and has GPIO pins can be used to replicate our 
setup. 
An introduction on the challenges of compiling and running CMIDID on the 
Raspberry Pi can be found in \texttt{andreas.pdf}
\subsubsection{Custom Keyboards}
\label{cmidid:keyboards}

Custom MIDI keyboard key events are simulated by interpreting input on the 
GPIO ports as key hits and key releases. This allows flexibility for the 
construction of custom hardware, but has the downside of increased hardware 
prototyping time and more involved driver configuration.

An overview of our sample keyboard setup on a breadboard can be found in \texttt{jannik.pdf}

\subsubsection{Module Configuration}
\label{cmidid:configuration}

The flexiblity of hardware development is probably the greatest advantage of 
the CMIDID module. To simply the process of MIDI keyboard construction, we 
added several ways to configure the module and the driver. It's possible to 
pass module parameters to define options which won't change during module 
runtime, like the number of available GPIO pins and the CMIDID driver can 
be configured during runtime via IOCTL from the userspace, which can be used 
for example to transpose the MIDI notes, i.e. add a constant to every note 
event.

The available configuration options are listed in \texttt{jannik.pdf}.

\subsubsection{Virtual MIDI device}
\label{cmidid:midi}

Besides the construction of the keyboard hardware, a virtual software based 
MIDI device is provided to communicate with other sound processing hardware 
like for example a software synthesize like FluidSynth 
(\url{http://www.fluidsynth.org}) or TiMidity++ 
(\url{http://timidity.sourceforge.net}). Fortunately, the Linux kernel 
offers an interface to easily handle MIDI devices: the ALSA sequencer API.
This API offers methods for MIDI 
device creation and manipulation and can be used from the userspace to 
route MIDI events between different devices.

Check out the the corresponding section to see how we used the ALSA sequencer API for
 this project: \texttt{andreas.pdf}

\subsubsection{GPIO Handling}
\label{cmidid:gpio}

The GPIO component of CMIDID module handles incoming triggers on the GPIO 
pins. Depending on the configuration, rising edge and falling edge events 
are interpreted as MIDI key hits and subsequent releases. Those signals are 
translated into a specific MIDI event: Either noteon or noteoff. Additionally, 
the GPIO component is responsible for the calculation of the velocity for 
the noteon events. This is done by measuring the time difference between 
the virtual key hits and interpolating the velocity according to a given 
interpolation function.

For more details on the handling of GPIO input, see \texttt{michael.pdf}.

\subsection{AppleMIDI}
\label{overview:applemidi}

\subsubsection{Author}

The AppleMIDI kernel driver was exclusively developed by Felix Engelmann.

\subsubsection{Information}

The AppleMIDI driver project was forked off the cmidid driver, as it might be useful for other projects too. The alsa sequencer core was considered a good interface, as it is low level enough for performance but easily controllable. Another advantage of two separate driver is, that when synthesizing directly on the local machine, there are no network security issues.
\\
The driver acts as a one-directional bridge for MIDI events from the alsa core to network clients.
For transport, it uses an extension of RTPMIDI (\href{http://tools.ietf.org/html/rfc6295}{RFC 6295})) which is called AppleMIDI. There, the session management is included into the RTPMIDI.
The foundation of the code is the midikit project, from which all libc independent functions were used.
\\
In the current implementation, there are several drawbacks from an optimal bridge. It does not accumulate multiple events which are issued in a short interval. Also only a subset of all MIDI commands is implemented for binary encoding. A backward correcting journal might be added too. The ideal functionality will provide both directions, so MIDI commands will be transmitted and received.
\\
For live performances, this module has an average processing time of several hundred microseconds, which is under the threshold of recognition.

\end{document}

