\documentclass[paper=a4,fontsize=11pt,pagesize,bibtotoc]{scrartcl}

\usepackage[utf8]{inputenc} 
\usepackage[T1]{fontenc}
\usepackage[english]{babel}

\usepackage[osf]{mathpazo} 
\usepackage{microtype}
\usepackage{tikz}
\usepackage{amsthm, amssymb, amsmath}
\usepackage{graphicx}
\usepackage{color}

\parindent0pt


\title{AppleMIDI}
\subtitle{Linux Kernel Driver Implementation}
\author{Felix Engelmann}

\begin{document}
	\maketitle
	
	As event based storage of music nearly disappeared completely from standard users, the 
	%TODO: full name aof midi abbreviation
	MIDI 
	is still the choice for digital instruments by professional musicians. The protocol is designed to transmit or store as many information as possible about the production of a tone. This normally includes the time at beginning of an event with the velocity of the beginning, which normally is related to the volume of the synthesized tone and a signal for the end, respectively the release of a key. Additionally to sound event transmission, there exists also a lot of control commands, e.g.:
	%TODO: list control commands of midi
	Depending on the command, it is encoded by 2 to 4 bytes.
	
	By the original standard, the signals are transmitted via a special 5 pin cable. With the wide acceptance of USB an implementation of MIDI over USB arose quickly because it replaced the need for a dedicated port and enabled an convenient way to plug multiple devices to one computer through a hub. But then there are still restrictions of USB. A better and also established model for transmission is an IP based network which can be on top of any hardware layer.
	
	To accomplish that, a group at the University of Berkeley wrote an RFC for MIDI over RTP. Our Work was to enable this possibility in the linux kernel to interconnect linux computer with midi links.
	
	\section{Related Work}
	Along with the RFC there exists a small reference implementation by 
	%TODO: Authors
	with the core functionality, but they advise to not use it as a library. But it provides valuable insight for complex parts.
	
	To enable RTPMIDI on Windows,
	%TODO: Author
	provides a driver. As it is a commercial product, it os closed source and therefore not helpful in any way.
	
	With the wide distribution of tablets it got more and more common to replace expensive MIDI-pads with virtual ones or any other ideas of apps with which it is possible to create sounds and send them over the network to a synthesizer. On Android, the nmj is a closed source java library.
	
	The apple operating systems also include applemidi as part of the coreaudio framework.
	
	%TODO: open source: nodejs rtpmidi, midikit
	
	There are also two open source implementations. 
	%TODO: nodejs author
	created an javascript version in nodejs. This has the advantage of high abstraction and provides a simple, easy to use interface. For building a bidirectional bridge, it is only required to implement receive hooks for each direction with one send cal to the other adapter.
	
	Im our test on the raspberry it served a good purpose as proof of concept to transfer midi over RTP. But due to the additional abstraction of the nodejs engine on the rather weak ARM, the processing time introduced a delay of about 6\,ms. This is too long for live performances.
	
	To get rid of the abstraction induced delay, 
	%TODO:midikit author
	implemented a universal adapter or alsa and RTP Midi directly in C. It is designed to be used in user-space applications with a run-loop to provide connectivity in non-blocking way. This is the code base of our driver, as all the management and book-keeping functions can be used as-is in the kernel.
	
	\section{RTPMIDI}
	The original as well as the updated RFC
	%TODO: cite
	are only specifying the transmission of MIDI commands as payload of a Real Time Protocol packet. this is transmitted within a UDP datagram. But it does not include a specification for the session management. It only refers to using some sort of session description protocol
	%TODO: cite SDP
	, e.g. SIP, which is also used to initiate phone and video calls over RTP.
	
	
	To minimize the divergence of combinations ant the resulting incompatibilities, Apple decided to use a standardized session session protocol. This is closely coupled to the RTPMidi transmission. Unfortunately the author decided to not publish the specification, however it found a broad acceptance. Because it is very simple, it was reverse engineered and there even exists a wireshark dissector to monitor the packets and their content.
	
	The state-machine of the session management is depicted in figure \ref{fig:state}
	
	
	The Applemidi communicates over two consecutive ports where an invitation and termination is required for both. The lower, normally even, default 5004, port is used for control commands. These include feedback, which provides the functionality of acknowledging received commands by the sequence number. The upper port is used for synchronization and the RTP connection.
	
	Parameters as the time-divisor,...
	%TODO: fixed parameters
	which are otherwise exchanged at session invitation are kept fix at predefined values.
	
	\section{Global Structure}
	
	As most of the code-base was ported from the midikit project, it dictated the global structure of the driver. As it should act like a bridge, it is connected to the sequencer core for receiving midi commands and provides a listening network socket, where peers can subscribe and receive data. There is a handler for each interface to accept and forward or process the received events.
	
	During initialization of the module, the sockets are set up and the connection as a virtual midi output device is created. In addition, internal data structures are allocated to store information about the driver itself, as well as a list of connected peers. All these structures are freed on removal. This includes the proper disconnection of all active peers.
	
	\subsection{Networking}
	
	Networking in the kernel can be done for multiple purposes. For our use-case it is only necessary to get a very height level access to the networking stack. 
	
	Handling with network clients normally does not require to be in kernel-space. Even further, running complex sever functionality in the kernel can introduce severe security issues. If there is a remote exploit for this module, there are no right barriers ant the attacker has immediately full access to the whole system. The wise spread apache server even forks a complete new process with its own address space to separate each request. With this, a compromised request can not influence or damage the complete system easily.
	This encapsulation is not possible in the kernel, as all modules run in the same memory space without any shielding. Nevertheless it is possible to create a server as a module.
	
	For a convenient way to interoperate with the networking stack, the netpoll api offers all necessary functionality. An other way to create a socket is by using the system calls which are also used from user-space. Concerning performance, all three options are orders of magnitude faster than the delay of the actual ethernet transmission over the link. But there exists cases, for example it there is no user-space management on very small embedded devices, where it is necessary to use the kernel version.
	
	
	
	
	
\end{document}