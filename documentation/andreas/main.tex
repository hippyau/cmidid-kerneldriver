\documentclass[paper=a4,fontsize=11pt,pagesize,bibtotoc]{scrartcl}

\usepackage[utf8]{inputenc} 
\usepackage[T1]{fontenc}
\usepackage[english]{babel}

\usepackage[osf]{mathpazo} 
\usepackage{microtype}
\usepackage{tikz}
\usepackage{amsthm, amssymb, amsmath}
\usepackage{graphicx}
\usepackage{color}

\usepackage{xr-hyper}
\usepackage{hyperref}

\parindent0pt


\title{cmidid documentation - alsa and gpio }
\subtitle{Linux Kernel Driver Implementation}
\author{Andreas R.}

\begin{document}
	\maketitle
	
This part of the documentation will provide information on the communication with the alsa sequencer and the interaction with the buttons of the keyboard connected over the GPIO Pins.
\section{Related Work}
Should we do related work in an documentation? This shouldn't be a scientific paper right?

\section{Alsa}
\label{alsa}
ALSA was initially developed by Jaroslav Kysela as replacement for the OSS/Free audio/midi subsystem which had a lack of features and didn't support most of the new sound cards. Since kernel version 2.6 ALSA is the default sound system used by the kernel.
\cite{Phillips:2005:UGA:1080072.1080075}
\bibliography{sources}{}
\bibliographystyle{plain}
	
	
\end{document}